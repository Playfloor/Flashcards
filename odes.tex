\documentclass[avery5371,grid]{flashcards}

%% Packages
\usepackage{amsmath}
\usepackage{amsfonts}
\usepackage{amssymb}
\usepackage{ccicons}
\usepackage[mathscr]{euscript}
\usepackage{hyperref}
\usepackage{url}

\hypersetup{
  pdftitle={Differential Equations and Linear Algebra Flash Cards},
  pdfsubject={Math},
  pdfauthor={Jason Underdown},
  pdfkeywords={differential equations, linear algebra}
}

%% Math macros
\newcommand{\N}{\mathbb{N}}
\newcommand{\Z}{\mathbb{Z}}
\newcommand{\Q}{\mathbb{Q}}
\newcommand{\R}{\mathbb{R}}
\newcommand{\C}{\mathbb{C}}
\newcommand{\B}{\mathscr{B}}
\newcommand{\st}{\textrm{ such that }}
%\renewcommand{\le}{\leqslant}
%\renewcommand{\theta}{\vartheta}
%\newcommand{\iso}{\cong}
\newcommand{\abs}[1]{\ensuremath{\left| #1 \right|}}
\newcommand{\set}[2]{\ensuremath{\left\{ #1 \, : \, #2 \right\}}}
%\newcommand{\presentation}[2]{\ensuremath{\left< #1 \, : \, #2 \right>}}
\newcommand{\normal}{\ensuremath{\lhd}}
\DeclareMathOperator{\Ker}{\ensuremath{\textrm{Ker}}}
\DeclareMathOperator{\Img}{\ensuremath{\textrm{Im}}}
%\DeclareMathOperator{\End}{\ensuremath{\textrm{End}}}

%% Text macros
\newcommand{\defn}[1]{\textbf{#1}}

%% Layout of flash cards
\cardfrontstyle[\large\slshape]{headings}
\cardbackstyle{empty}
\cardfrontfoot{ODEs and Linear Algebra}

\begin{document}

%%%%%%%%%%%%%%%%%%%%%%%%%%%%%%%%%%%%%%%%%%%%%%%%%%%%%%%%%%%%%%%%%%%%%%%%%%%%%%%
\begin{flashcard}[Copying]
  { Flash Cards for Math 2250

    \begin{center}
      ``Differential Equations and Linear Algebra''
    \end{center}
  }
  \vspace*{\stretch{1}}

  \copyright\  2017 Jason Underdown \\ \\
  This work is licensed under a \\
  Creative Commons Attribution 4.0 \\
  International License \\
  \ccby \\ \\
  \url{https://creativecommons.org/licenses/by/4.0/}

  \vspace*{\stretch{1}}
\end{flashcard}

\begin{flashcard}[Definition]{order}
  \vspace*{\stretch{1}}

  The \defn{order} of a differential equation (DE) is the order of the
  highest derivative which occurs in the equation.

  \vspace*{\stretch{1}}
\end{flashcard}

\begin{flashcard}[Definition]{ODE and PDE}
  \vspace*{\stretch{1}}

  An \defn{ODE} (ordinary differential equation) is a DE that only
  contains total derivatives.
  \bigskip

  A \defn{PDE} (partial differential equation) is a DE which contains
  at least one partial derivative.

  \vspace*{\stretch{1}}
\end{flashcard}

\begin{flashcard}[Definition]{directly integrable}
  \vspace*{\stretch{1}}

  The DE, $y' = F(x,y)$, is \defn{directly integrable} if
  \[
    F(x,y)=f(x).
  \]
  Such an equation may be solved by computing:
  \[
    y(x) = \int f(x) \, dx.
  \]

  \vspace*{\stretch{1}}
\end{flashcard}

\begin{flashcard}[Definition]{general solution}
  \vspace*{\stretch{1}}

  A \defn{general solution} to a DE is any continuous function which
  satisfies the DE and contains at least one constant of integration.

  \bigskip

  A general solution is actually a collection or \emph{family} of
  functions parametrized by the integration constant(s).

  \vspace*{\stretch{1}}
\end{flashcard}

\begin{flashcard}[Definition]{initial value problem \\(IVP)}
  \vspace*{\stretch{1}}

  An \defn{initial value problem} or \defn{IVP} is a differential
  equation coupled with at least one initial condition, for example,
  \[
    y' = F(x,y) \qquad y(0) = y_0.
  \]

  A second order IVP will require two initial conditions, $y(0)=y_0$
  and $y'(0)=v_0$. In general, an $n$th order IVP will require $n$
  initial conditions.

  \vspace*{\stretch{1}}
\end{flashcard}

\begin{flashcard}[Definition]{particular solution}
  \vspace*{\stretch{1}}

  A \defn{particular solution} is a solution to an IVP. In other words
  it is a single function which satisfies both the DE and any initial
  condition(s).

  \vspace*{\stretch{1}}
\end{flashcard}

\begin{flashcard}[Algorithm]{slope field plot}
  \vspace*{\stretch{1}}

  Given a first--order DE, $y' = F(x,y)$, one can graphically
  approximate solutions to the DE by generating a \defn{slope field
    plot}:
  \begin{enumerate}
  \item Divide a region of the $xy$--plane into a grid.
  \item For each cell in the grid compute $m = F(\bar{x}, \bar{y})$,
    where $(\bar{x}, \bar{y})$ is the midpoint of the cell, and plot a
    small bar with slope $m$ centered on the point
    $(\bar{x}, \bar{y})$.
  \end{enumerate}

  \vspace*{\stretch{1}}
\end{flashcard}

\begin{flashcard}[Theorem]{existence}
  \vspace*{\stretch{1}}

  \begin{equation}
    \label{eq:first-order-ivp}
    y' = F(x,y) \qquad y(a) = b
  \end{equation}

  \textbf{If} $F(x,y)$ is continuous on some rectangle $R$ containing
  the point $(a,b)$ in its interior, \textbf{then} there exists an
  open interval $I$ which contains $a$ such that
  \eqref{eq:first-order-ivp} has a solution on $I$. Note that the
  width of $I$ may be shorter than the width of $R$.

  \vspace*{\stretch{1}}
\end{flashcard}

\begin{flashcard}[Theorem]{uniqueness}
  \vspace*{\stretch{1}}

  \begin{equation}
    \tag{\ref{eq:first-order-ivp}}
    y' = F(x,y) \qquad y(a) = b
  \end{equation}

  \textbf{If} $F(x,y)$ and $F_y(x,y)$ are both continuous on some
  rectangle $R$ containing the point $(a,b)$ in its interior,
  \textbf{then} there exists an open interval $I$ which contains $a$
  such that \eqref{eq:first-order-ivp} has a \emph{unique} solution on
  $I$. Note that the width of $I$ may be shorter than the width of
  $R$.

  \vspace*{\stretch{1}}
\end{flashcard}

\begin{flashcard}[Definition]{separable}
  \vspace*{\stretch{1}}

  The DE $y' = F(x,y)$ is called \defn{separable} if the right hand
  side function can be expressed as a product of two single variable
  functions, for example,
  \[
    F(x,y) = f(x)g(y).
  \]
  Such equations may be solved by ``separating the variables'', that
  is, computing:
  \[
    \int \frac{1}{g(y)} \, dy = \int f(x) \, dx.
  \]

  \vspace*{\stretch{1}}
\end{flashcard}

\begin{flashcard}[Definition]{population model}
  \vspace*{\stretch{1}}

  The \defn{population model} describes exponential growth and decay. It
  is an IVP:
  \[
    P' = kP \qquad P(0) = P_0.
  \]
  This model has particular solution:
  \[
    P(t) = P_0 e^{kt}.
  \]
  Using the particular solution to extrapolate values may not be
  reasonable because exponential growth is often unphysical
  (impossible) over a long time scale.

  \vspace*{\stretch{1}}
\end{flashcard}

\begin{flashcard}[Definition]{diffusion model \\ (Newton cooling and heating)}
  \vspace*{\stretch{1}}

  The \defn{diffusion model} describes how something diffuses or
  spreads into its \emph{ambient} environment.
  \[
    y' = k(A-y) \qquad y(0) = y_0
  \]
  This model has solution,
  \[
    y(t) = A - Ce^{-kt} \qquad C = A-y_0.
  \]
  Do not try to memorize the solution! Solve the model via separation
  of variables.

  \vspace*{\stretch{1}}
\end{flashcard}

\begin{flashcard}[Definition]{linear DE}
  \vspace*{\stretch{1}}

  An ODE is \defn{linear} if it can be written in the form:
  \[
    a_n(x)y^{(n)} + a_{n-1}(x)y^{(n-1)} + \cdots + a_1(x)y' + a_0(x)y
    = f(x).
  \]
  Example of first and second order, \emph{linear}  ODEs:
  \begin{align*}
    a_1(x)y' + a_0(x)y &= f(x) \\
    a_2(x)y'' + a_1(x)y' + a_0(x)y &= f(x)
  \end{align*}

  \vspace*{\stretch{1}}
\end{flashcard}

\begin{flashcard}[Definition]{integrating factor}
  \vspace*{\stretch{1}}

  Given a linear, first order ODE:
  \[
    a_1(x)y' + a_0(x)y = f(x),
  \]
  first divide by $a_1(x)$ to yield:
  \[
    y' + p(x)y = q(x).
  \]
  The \defn{integrating factor} is: \( I(x) = e^{\int \! p(x) dx}. \)

  \vspace*{\stretch{1}}
\end{flashcard}

\begin{flashcard}[Algorithm]{integrating factor method}
  \vspace*{\stretch{1}}

  \begin{enumerate}
  \item Put in standard form:
    \(
      y' + p(x)y = q(x).
    \)
  \item Multiply both sides by $I(x) = e^{\int\! p(x)dx}$, \\
    and integrate:

    \(
    y I(x) = \displaystyle \int q(x)I(x) \, dx
    \)
  \item
    \(
    y = \displaystyle e^{-\int\! p(x) dx}
    \left[ \int \left( q(x)e^{\int \!p(x) dx} \right)\, dx \right]
    \)

    Don't forget the $+C$ when integrating!

  \end{enumerate}




  \vspace*{\stretch{1}}
\end{flashcard}

\begin{flashcard}[Definition]{autonomous}
  \vspace*{\stretch{1}}

  The DE $y' = F(x,y)$ is called \defn{autonomous} if $F(x,y)$ is just
  a function of the dependent variable $y$, that is if:
  \[
    F(x,y) = g(y).
  \]
  Such equations can be analyzed by finding the roots of the equation
  $g(y)=0$ and then creating a phase diagram.

  \vspace*{\stretch{1}}
\end{flashcard}


% \begin{flashcard}[Definition]{vector space}
%   \vspace*{\stretch{1}}

%   A \defn{vector space} over a field $F$ is a set $V$, equipped with
%   addition and scalar multiplication satisfying:
%   \begin{enumerate}
%   \item $V$ is an abelian group under addition;
%   \item $\forall \, u,v \in V$ and $\forall \, \lambda, \mu \in F$,
%     \begin{enumerate}
%     \item $\lambda(u + v) = \lambda u + \lambda v$
%     \item $(\lambda + \mu)v = \lambda v + \mu v$
%     \item $(\lambda \mu) v = \lambda (\mu v)$
%     \item $1 v = v$
%     \end{enumerate}
%   \end{enumerate}

%   \vspace*{\stretch{1}}
% \end{flashcard}

% \begin{flashcard}[Definition]{linear dependence / linear independence}
%   \vspace*{\stretch{1}}

%   We say that $v_1, \ldots, v_n$ are \defn{linearly dependent} if
%   \[
%     \lambda_1 v_1 + \cdots + \lambda_n v_n = 0
%   \]
%   for some $\lambda_1, \ldots, \lambda_n \in F$ not all zero,
%   otherwise the vectors $v_1, \ldots, v_n$ are \defn{linearly
%     independent}. \vspace*{\stretch{1}}
% \end{flashcard}

% \begin{flashcard}[Definition]{linear combination / span}
%   \vspace*{\stretch{1}}

%   Let $v_1, \ldots, v_n$ be vectors in a vector space $V$ over $F$. A
%   vector $v$ in $V$ is a \defn{linear combination} of
%   $v_1, \ldots, v_n$ if
%   \[
%     v = \lambda_1 v_1 + \cdots + \lambda_n v_n
%   \]
%   for some $\lambda_1, \ldots, \lambda_n \in F$.
%   \vfill

%   The vectors $v_1, \ldots, v_n$ \defn{span} $V$ if every vector in
%   $V$ is a linear combination of $v_1, \ldots, v_n$.

%   \vspace*{\stretch{1}}
% \end{flashcard}

% \begin{flashcard}[Definition]{basis}
%   \vspace*{\stretch{1}}

%   The vectors $v_1, \ldots , v_n \in V$ form a \defn{basis} of V if
%   they
%   \begin{enumerate}
%   \item \emph{span} V, and are
%   \item \emph{linearly independent}.
%   \end{enumerate}

%   \vspace*{\stretch{1}}
% \end{flashcard}

% \begin{flashcard}[Definition/Theorem]{subspace / conditions for a subspace}
%   \vspace*{\stretch{1}}

%   A \defn{subspace} of a vector space $V$ over $F$ is a subset of $V$
%   which is itself a vector space under the operations inherited from
%   $V$.
%   \vfill

%   A subset $U$ of a vector space $V$ is a subspace iff
%   \begin{enumerate}
%   \item $0\in U$;
%   \item if $u,v \in U$ then $u+v \in U$;
%   \item if $\lambda \in F$ and $u \in U$ then $\lambda u \in U$.
%   \end{enumerate}

%   \vspace*{\stretch{1}}
% \end{flashcard}

% \begin{flashcard}[Definition]{linear transformation}
%   \vspace*{\stretch{1}}

%   Let $V$ and $W$ be vector spaces over $F$. A \defn{linear
%     transformation} from $V$ to $W$ is a function
%   \[
%     \theta : V \to W
%   \]
%   which satisfies
%   \begin{enumerate}
%   \item $(u + v)\theta = u\theta + v\theta$ for all $u,v \in V$, and
%   \item $(\lambda u)\theta = \lambda (v\theta)$ for all
%     $\lambda \in F$ and $v \in V$.
%   \end{enumerate}

%   \vspace*{\stretch{1}}
% \end{flashcard}

% \begin{flashcard}[Theorem]{rank--nullity theorem}
%   \vspace*{\stretch{1}}

%   Suppose $V$ and $W$ are vector spaces and
%   \[
%     \theta : V \to W
%   \]
%   is a linear transformation, then
%   \[
%     \dim V = \dim(\Ker \theta) + \dim(\Img \theta)
%   \]

%   \vspace*{\stretch{1}}
% \end{flashcard}

% \begin{flashcard}[Theorem]{invertibility of linear transformations}
%   \vspace*{\stretch{1}}

%   Let $\theta$ be a linear transformation from $V$ to itself, then the
%   following conditions are equivalent:
%   \begin{enumerate}
%   \item $\theta$ is invertible,
%   \item $\Ker \theta = \{ 0 \}$,
%   \item $\Img \theta  = V$.
%   \end{enumerate}

%   \vspace*{\stretch{1}}
% \end{flashcard}


% \begin{flashcard}[Definition]{matrix of an endomorphism \\ $[\theta ]_{\B}$}
%   \vspace*{\stretch{1}}

%   Let $V$ be a vector space over $F$, and let $\theta$ be an
%   endomorphism of $V$. Once a basis $\B = \{ v_1, \ldots, v_n \}$ for
%   $V$ is chosen, then there are $n^2$ scalars
%   $a_{ij} \in F \; (1 \le i,j \le n)$ such that for all $i$:
%   \[
%     v_i \theta = a_{i1}v_1 + \cdots + a_{in}v_n.
%   \]
%   The $n\times n$ matrix $(a_{ij})$ is called the \defn{matrix of
%     $\theta$ relative to the basis $\B$}, and is denoted by
%   $[\theta ]_{\B}$.

%   \vspace*{\stretch{1}}
% \end{flashcard}


% \begin{flashcard}[Definition]{change of basis matrix}
%   \vspace*{\stretch{1}}

%   Let $\B = \{v_1, \ldots, v_n \}$ be a basis of the vector space V,
%   and let $\B' = \{v'_1, \ldots, v'_n \}$ be another basis of $V$.
%   Then for $1 \le i \le n$,
%   \[
%     v'_i = t_{i1} v_1 + \cdots + t_{in} v_n
%   \]
%   for certain scalars $t_{ij}$. The $n\times n$ matrix $T=(t_{ij})$ is
%   invertible and is called the \defn{change of basis matrix} from $\B$
%   to $\B'$.

%   \vspace*{\stretch{1}}
% \end{flashcard}

% \begin{flashcard}[Theorem]{change of basis}
%   \vspace*{\stretch{1}}

%   If $\B$ and $\B'$ are bases of $V$ and $\theta$ is an endomorphism
%   of $V$, then
%   \[
%     [\theta]_{\B} = T^{-1}[\theta]_{\B'}T,
%   \]
%   where $T$ is the change of basis matrix from $\B$ to $\B'$.

%   \vspace*{\stretch{1}}
% \end{flashcard}

\end{document}
