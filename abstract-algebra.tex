\documentclass[avery5371,grid]{flashcards}

\cardfrontstyle[\large\slshape]{headings}
\cardbackstyle{empty}
\cardfrontfoot{Abstract Algebra I}

%\usepackage{amsfonts}
\usepackage{amssymb,amsmath}

\begin{document}
\input{tex.macros}

\begin{flashcard}[Copyright \& License]{Copyright \copyright \, 2006 Jason Underdown \\
Some rights reserved.}
\vspace*{\stretch{1}}
These flashcards and the accompanying \LaTeX \, source code are licensed
under a Creative Commons Attribution--NonCommercial--ShareAlike 2.5 License.  
For more information, see creativecommons.org.  You can contact the author at:
\begin{center}
\begin{small}\tt jasonu [remove-this] at physics dot utah dot edu\end{small}
\end{center}
\vspace*{\stretch{1}}
\end{flashcard}

\begin{flashcard}[Definition]{set operations}
\vspace*{\stretch{1}}
\begin{eqnarray*}
A \cup B &=& \lbrace x\mid x\in A \;\textrm{or}\; x\in B\rbrace \\
A \cap B &=& \lbrace x\mid x\in A \;\textrm{and}\; x\in B\rbrace \\
A - B &=& \lbrace x\mid x\in A \;\textrm{and}\; x\not\in B\rbrace \\
A + B &=& (A-B) \cup (B-A)\\ 
\end{eqnarray*}
\vspace*{\stretch{1}}
\end{flashcard}

\begin{flashcard}[Theorem]{De Morgan's rules}
\vspace*{\stretch{1}}
For $A$, $B \subseteq S$
\begin{eqnarray*}
(A\cap B)' &=& A' \cup B'\\
(A\cup B)' &=& A' \cap B'
\end{eqnarray*}
\vspace*{\stretch{1}}
\end{flashcard}

\begin{flashcard}[Definition]{surjective or onto mapping}
\vspace*{\stretch{1}}
The mapping $f:S \mapsto T$ is \textit{onto} or \textit{surjective} if every 
$t \in T$ is the image under $f$ of some $s \in S$; that is, iff, $\forall \; t \in T, \quad \exists \; s \in S$ such that $t = f(s)$.
\vspace*{\stretch{1}}
\end{flashcard}

\begin{flashcard}[Definition]{injective or one--to--one mapping}
\vspace*{\stretch{1}}
The mapping $f:S \mapsto T$ is \textit{injective} or \textit{one--to--one} 
\mbox{(1-1)} if for $s_1 \neq s_2$ in $S$, $f(s_1) \neq f(s_2)$ in $T$.

\medskip
Equivalently:
\begin{equation*}
f\; \mathrm{ injective } \Longleftrightarrow f(s_1) = f(s_2) \Rightarrow s_1 = s_2
\end{equation*}
\vspace*{\stretch{1}}
\end{flashcard}

\begin{flashcard}[Definition]{bijection}
\vspace*{\stretch{1}}
The mapping $f:S \mapsto T$ is said to be a \textit{bijection} if $f$ is both 
\mbox{1-1} and onto.
\vspace*{\stretch{1}}
\end{flashcard}

\begin{flashcard}[Definition]{composition of functions}
\vspace*{\stretch{1}}
Suppose $g: S \mapsto T$ and $f: T \mapsto U$, then the \textit{composition}
or \textit{product}, denoted by $f\circ g$ is the mapping $f\circ g: S \mapsto U$
defined by:
\begin{equation*}
(f\circ g)(s) = f(g(s))
\end{equation*}
\vspace*{\stretch{1}}
\end{flashcard}


\begin{flashcard}[Lemma]{composition of functions is associative}
\vspace*{\stretch{1}}
If $h: S \mapsto T, g:T \mapsto U$, and $f: U \mapsto V$, then,
\begin{equation*}
f\circ(g\circ h) = (f\circ g)\circ h
\end{equation*}
\vspace*{\stretch{1}}
\end{flashcard}

\begin{flashcard}[Lemma]{cancellation and composition}
\vspace*{\stretch{1}}
\begin{equation*}
f \circ g = f \,\circ \stackrel{\sim}{g} \; \textrm{and} \; f\;\textrm{is 1--1}
\Rightarrow g = \,\stackrel{\sim}{g}
\end{equation*}
\begin{equation*}
f \circ g = \,\stackrel{\sim}{f} \circ\, g\; \textrm{and $g$ is onto} \;
\Rightarrow f = \,\stackrel{\sim}{f}
\end{equation*}
\vspace*{\stretch{1}}
\end{flashcard}

\begin{flashcard}[Definition]{image and inverse image of a function}
\vspace*{\stretch{1}}
Suppose $f: S\mapsto T$, and $U\subseteq S$, then the \textit{image}
of $U$ under $f$ is
\begin{equation*}
f(U)=\lbrace f(u)\mid u\in U\rbrace
\end{equation*}

\bigskip
If $V\subseteq T$ then the \textit{inverse image} of $V$ under $f$ is
\begin{equation*}
f^{-1}(V) = \lbrace s\in S \mid f(s) \in V\rbrace
\end{equation*}
\vspace*{\stretch{1}}
\end{flashcard}

\begin{flashcard}[Definition]{inverse function}
\vspace*{\stretch{1}}
Suppose $f: S\mapsto T$.  An \textit{inverse} to $f$ is a function
$f^{-1}:T\mapsto S$ such that
\begin{eqnarray*}
f\circ f^{-1} &=& i_T\\
f^{-1}\circ f &=& i_S
\end{eqnarray*}
Where $i_T:T\mapsto T$ is defined by $i_T(t) = t$, and is called the
\textit{identity function} on $T$.  And similarly for $S$. 
\vspace*{\stretch{1}}
\end{flashcard}

\begin{flashcard}[Definition]{$A(S)$}
\vspace*{\stretch{1}}
If $S$ is a nonempty set, then $A(S)$ is the set of all 1--1 mappings of $S$
onto itself.

\medskip
When $S$ has a finite number of elements, say $n$, then $A(S)$ is called the
\textit{symmetric group of degree n} and is often denoted by $S_n$.
\vspace*{\stretch{1}}
\end{flashcard}

\begin{flashcard}[Lemma]{properties of $A(S)$}
\vspace*{\stretch{1}}
$A(S)$ satisfies the following:
\begin{enumerate}
\item $f,g \in A(S)\Rightarrow f\circ g \in A(S)$
\item $f,g,h \in A(S)\Rightarrow (f\circ g)\circ h = f\circ (g\circ h)$
\item There exists an $i$ such that $f\circ i = i\circ f = f$\
$\forall f \in A(S)$
\item Given $f \in A(S)$, there exists a $g \in A(S)$ such~that 
$f\circ g = g\circ f = i$ 
\end{enumerate}
\vspace*{\stretch{1}}
\end{flashcard}

\begin{flashcard}[Definition]{group}
\vspace*{\stretch{1}}
A nonempty set $G$ together with some operator $*$ is said to be a \textit{group} if:
\begin{enumerate}
\item If $a, b \in G$ then $a*b \in G$
\item If $a, b, c \in G$ then $a*(b*c) = (a*b)*c$
\item $G$ has an identity element $e$ such that 
\mbox{$a*e=e*a=a \;\; \forall \> a \in G$}
\item $\forall \> a \in G, \;\; \exists \> b \in G $ such that
\mbox{$a*b=b*a=e$}
\end{enumerate}
\vspace*{\stretch{1}}
\end{flashcard}

\begin{flashcard}[Definition]{order of a group}
\vspace*{\stretch{1}}
The number of elements in $G$ is called the \textit{order} of $G$ and 
is denoted by $\vert G \vert$.
\vspace*{\stretch{1}}
\end{flashcard}

\begin{flashcard}[Definition]{abelian}
\vspace*{\stretch{1}}
A group $G$ is said to be \textit{abelian} if $\quad \forall \; a,b\in G$
\begin{equation*}
a*b=b*a
\end{equation*}
\vspace*{\stretch{1}}
\end{flashcard}

\begin{flashcard}[Lemma]{properties of groups}
\vspace*{\stretch{1}}
If $G$ is a group then
\begin{enumerate}
\item Its identity element, $e$ is unique.
\item Every $a \in G$ has a unique inverse $a^{-1} \in G$.
\item If $a \in G$, then $(a^{-1})^{-1}=a$.
\item For $a,b \in G$, \mbox{$(ab)^{-1} = b^{-1}a^{-1}$}, where
$ab=a*b$.
\end{enumerate}
\vspace*{\stretch{1}}
\end{flashcard}

\begin{flashcard}[Definition]{subgroup}
\vspace*{\stretch{1}}
A nonempty subset, $H$ of a group $G$ is called a \mbox{\textit{subgroup}}
of $G$ if, relative to the operator in $G$, $H$ itself forms a group.
\vspace*{\stretch{1}}
\end{flashcard}

\begin{flashcard}[Lemma]{when is a subset a subgroup}
\vspace*{\stretch{1}}
A nonempty subset $A\subset G$ is a subgroup $\Leftrightarrow$ \
$A$ is closed with respect to the operator of $G$ and given $a \in A$ then $a^{-1} \in A$.
\vspace*{\stretch{1}}
\end{flashcard}

\begin{flashcard}[Definition]{cyclic subgroup}
\vspace*{\stretch{1}}
A \textit{cyclic subgroup} of $G$ is generated by a single element $a \in G$ and is denoted by $(a)$.
\begin{equation*}
(a) = \left\lbrace a^i \mid i \; \textrm{any integer} \right\rbrace
\end{equation*}
\vspace*{\stretch{1}}
\end{flashcard}

\begin{flashcard}[Lemma]{finite subsets and subgroups}
\vspace*{\stretch{1}}
Suppose that $G$ is a group and $H$ a nonempty \textit{finite} subset of $G$
closed under the operation in $G$.  Then $H$ is a subgroup of $G$.

\textbf{Corollary}  If $G$ is a \textit{finite} group and $H$ a nonempty subset of $G$
closed under the operation of $G$, then $H$ is a subgroup of $G$.
\vspace*{\stretch{1}}
\end{flashcard}

\begin{flashcard}[Lemma]{subgroups under $\cap$ and $\cup$}
\vspace*{\stretch{1}}
Suppose $H$ and $H'$ are subgroups of $G$, then
\begin{itemize}
\item $H \cap H'$ is a subgroup of $G$
\item $H \cup H'$ is \textbf{not} a subgroup of $G$, as long as neither
$H$ nor $H'$ is contained in the other.
\end{itemize}
\vspace*{\stretch{1}}
\end{flashcard}

\begin{flashcard}[Definition]{equivalence relation}
\vspace*{\stretch{1}}
A relation $\sim$ on elements of a set $S$ is an \textit{equivalence relation}
if for all $a, b, c \in S$ it satisfies the following criteria:
\begin{enumerate}
\item $a \sim a$ reflexivity
\item $a \sim b \Rightarrow b \sim a$ symmetry
\item $a \sim b$ and $b \sim c \Rightarrow a \sim c$ transitivity
\end{enumerate}
\vspace*{\stretch{1}}
\end{flashcard}

\begin{flashcard}[Definition]{equivalence class}
\vspace*{\stretch{1}}
If $\sim$ is an equivalence relation on a set $S$, then the
\textit{equivalence class} of $a$ denoted $[a]$ is defined to be:
\begin{equation*}
[a] = \left\lbrace b \in S \mid b \sim a\right\rbrace 
\end{equation*}
\vspace*{\stretch{1}}
\end{flashcard}

\begin{flashcard}[Theorem]{equivalence relations partition sets}
\vspace*{\stretch{1}}
If $\sim$ is an equivalence relation on a set $S$, then $\sim$ partitions $S$
into equivalence classes.  That is, for any $a, b \in S$ either:
\begin{equation*}
[a] = [b] \quad \textrm{or} \quad [a] \cap [b] = \oslash
\end{equation*}
\vspace*{\stretch{1}}
\end{flashcard}

\begin{flashcard}[Theorem]{Lagrange's theorem}
\vspace*{\stretch{1}}
If $G$ is a finite group and $H$ is a subgroup of $G$, then the order of $H$ divides the order of $G$.  That is,
\begin{equation*}
\left| G \right| = k \left| H \right|
\end{equation*}
for some integer $k$.  The converse of Lagrange's theorem is not generally true.
\vspace*{\stretch{1}}
\end{flashcard}

\begin{flashcard}[Definition]{index of a subgroup}
\vspace*{\stretch{1}}
If $G$ is a finite group, and $H$ a subgroup of $G$, then the
\textit{index} of $H$ in $G$ is the number of distinct right cosets
of $H$ in $G$, and is denoted:
\begin{equation*}
[G : H] = \dfrac{|G|}{|H|} = i_{G}(H)
\end{equation*}
\vspace*{\stretch{1}}
\end{flashcard}

\begin{flashcard}[Definition]{order of an element in a group}
\vspace*{\stretch{1}}
If $a$ is an element of $G$ then the \textit{order} of $a$ denoted by $o(a)$ is the least positive integer $m$ such that $a^m = e$.
\vspace*{\stretch{1}}
\end{flashcard}

\begin{flashcard}[Theorem]{finite groups wrap around}
\vspace*{\stretch{1}}
If $G$ is a finite group of order $n$ then $a^n = e$ for all $a \in G$.
\vspace*{\stretch{1}}
\end{flashcard}

\Card{Abelian group}{
    AKA a commutative group:
    a group whose operation is also commutative: $a\cdot b = b\cdot a$.
}

\Card{Monoid}{
    A set with an operation that is closed and associative,
    and an identity element where $f(x)=x$.

    I.e., a semigroup with an identity.
}

\Card{Ring}{
    A set $R$ with two binary operations, annotated $+$ and $\cdot$, where
    \items{
        \item $\{R, +\}$ is an abelian (commutative) group
        \item $\{R, \cdot\}$ is a monoid
        \item The distributive property holds:
            $$a \cdot (b + c) = (a \cdot b) + (a \cdot c)\\
            (b + c) \cdot a = (b \cdot a) + (c \cdot a) $$
    }
}

\begin{flashcard}[Definition]{homomorphism}
\vspace*{\stretch{1}}
If $G$ and $G'$ are two groups, then the mapping
\begin{equation*}
f: G\rightarrow G'
\end{equation*}
is a \textit{homomorphism} if 
\begin{equation*}
f(ab) = f(a)f(b) \qquad \forall \; a,b \in G
\end{equation*}
\vspace*{\stretch{1}}
\end{flashcard}

\begin{flashcard}[Definition]{monomorphism, isomorphism, automorphism}
\vspace*{\stretch{1}}
\begin{small}
Suppose the mapping $f: G\rightarrow G'$ is a homomorphism, then:
\begin{itemize}
\item If $f$ is 1--1 (injective) it is called a \textit{monomorphism}.
\item If $f$ is 1--1 and onto (injective and surjective), then it is called an \mbox{\textit{isomorphism}}.
\item If $f$ is an isomorphism that maps $G$ onto itself then it is called
an \textit{automorphism}.
\item If an isomorphism exists between two groups then they are said to be
\textit{isomorphic} and denoted $G\simeq G'$.
\end{itemize}
\end{small}
\vspace*{\stretch{1}}
\end{flashcard}

\begin{flashcard}[Theorem]{composition of homomorphisms}
\vspace*{\stretch{1}}
Suppose $f : G \mapsto G'$ and $h : G' \mapsto G''$ are homomorphisms,
then the composition of $h$ with $f$, $h\circ f$ is also a homomorphism.
\vspace*{\stretch{1}}
\end{flashcard}

\begin{flashcard}[Definition]{kernel}
\vspace*{\stretch{1}}
If $f$ is a homomorphism from $G$ to $G'$ (which has identity $e'$) then the \textit{kernel}
of $f$ is
\begin{equation*}
\textrm{Ker}f \equiv \lbrace a \in G \mid f(a)=e' \rbrace
\end{equation*}
\vspace*{\stretch{1}}
\end{flashcard}

\begin{flashcard}[Theorem]{kernel related subgroups}
\vspace*{\stretch{1}}
If $f$ is a homomorphism of $G$ into $G'$, then
\begin{enumerate}
\item $\textrm{Ker} f$ is a subgroup of $G$.
\item If $a \in G$ then $a^{-1}(\textrm{Ker} f) a \subset \textrm{Ker} f$.
\end{enumerate}
\vspace*{\stretch{1}}
\end{flashcard}

\begin{flashcard}[Definition]{normal subgroup}
\vspace*{\stretch{1}}
A subgroup $N$ of $G$ is said to be a \textit{normal subgroup}
of $G$ if $a^{-1}Na \subset N$ for each $a \in G$.
\medskip
\\
$N$ normal to $G$ is denoted $N \lhd G$.
\vspace*{\stretch{1}}
\end{flashcard}

\begin{flashcard}[Theorem]{normal subgroups and their cosets}
\vspace*{\stretch{1}}
$N \lhd G$ iff every left coset of $N$ in $G$ is also a right
coset of $N$ in $G$.
\vspace*{\stretch{1}}
\end{flashcard}

\begin{flashcard}[Definition/Theorem]{factor group}
\vspace*{\stretch{1}}
If $N \lhd G$, then we define the \textit{factor group}
of $G$ by $N$ denoted $G/N$ to be:
\begin{equation*}
G/N = \lbrace Na \mid a \in G \rbrace = 
\lbrace [a] \mid a \in G \rbrace
\end{equation*}
\medskip
$G/N$ is a group relative to the operation
\begin{equation*}
(Na)(Nb) = Nab
\end{equation*}
\vspace*{\stretch{1}}
\end{flashcard}

\begin{flashcard}[Theorem]{normal subgroups are the kernel of a homomorphism}
\vspace*{\stretch{1}}
If $N \lhd G$, then there is a homomorphism $\psi : G \mapsto G/N$
such that $\textrm{Ker} \psi = N$.
\vspace*{\stretch{1}}
\end{flashcard}

\begin{flashcard}[Theorem]{order of a factor group}
\vspace*{\stretch{1}}
If $G$ is a finite group and $N \lhd G$, then
\begin{equation*}
\left| G/N \right| = \frac{\left| G \right|}{\left| N \right|}
\end{equation*}
\vspace*{\stretch{1}}
\end{flashcard}

\begin{flashcard}[Theorem]{Cauchy's theorem}
\vspace*{\stretch{1}}
If $p$ is a prime that divides $\left| G \right|$,
then $G$ has an element of order $p$.
\vspace*{\stretch{1}}
\end{flashcard}

\begin{flashcard}[Theorem]{first homomorphism theorem}
\vspace*{\stretch{1}}
If $\varphi : G \mapsto G'$ is an onto homomorphism with kernel $K$ then,
\begin{equation*}
G/K \simeq G'
\end{equation*}
with isomorphism $\psi : G/K \mapsto G'$ defined by
\begin{equation*}
\psi (Ka) = \varphi(a)
\end{equation*}
\vspace*{\stretch{1}}
\end{flashcard}

\begin{flashcard}[Theorem]{correspondence theorem}
\vspace*{\stretch{1}}
Let $\varphi : G \mapsto G'$ be a homomorphism which maps $G$
onto $G'$ with kernel $K$.  If $H'$ is a subgroup of $G'$, and if
$H' = \lbrace a \in G \mid \varphi (a) \in H' \rbrace$ then
\begin{itemize}
\item $H$ is a subgroup of $G$
\item $K \subset H$
\item $H/K \simeq H'$
\end{itemize}
Also, if $H' \lhd G'$ then $H \lhd G$.
\vspace*{\stretch{1}}
\end{flashcard}

\begin{flashcard}[Theorem]{second isomorphism theorem}
\vspace*{\stretch{1}}
Let $H$ be a subgroup of $G$ and $N \lhd G$, then
\begin{enumerate}
\item $HN = \lbrace hn \mid h\in H, n\in N \rbrace$ is a subgroup of $G$
\item $H\cap N \lhd H$
\item $H/(H\cap N) \simeq (HN)/N$
\end{enumerate}
\vspace*{\stretch{1}}
\end{flashcard}

\begin{flashcard}[Theorem]{third isomorphism theorem}
\vspace*{\stretch{1}}
If $\varphi : G \mapsto G'$ is an onto homomorphism with kernel $K$ and
if $N' \lhd G'$ with 
$N = \lbrace a \in G \mid \varphi (a) \in N' \rbrace$ then
\begin{equation*}
G/N \simeq G'/N'
\end{equation*}
\begin{center}
or equivalently
\end{center}
\begin{equation*}
G/N \simeq \frac{G/K}{N/K}
\end{equation*}
\vspace*{\stretch{1}}
\end{flashcard}

\begin{flashcard}[Theorem]{groups of order $pq$}
\vspace*{\stretch{1}}
If $G$ is a group of order $pq$ ($p$ and $q$ primes) where
$p > q$ and $q \not{\mid} \; p-1$ then $G$ must be cyclic.
\vspace*{\stretch{1}}
\end{flashcard}

\begin{flashcard}[Definition]{external direct product}
\vspace*{\stretch{1}}
Suppose $G_{1},\ldots,G_{n}$ is a collection of groups.  The
\textit{external direct product} of these $n$ groups is the set
of all $n$--tuples for which the $i$th component is an element of
$G_{i}$.
\begin{equation*}
G_{1}\times G_{2}\times \ldots\times G_{n} =
\left\lbrace \left( g_{1},g_{2},\ldots,g_{n}\right)
\mid g_{i} \in G_{i}\right\rbrace 
\end{equation*}
The product is defined component--wise.
\begin{equation*}
\left(a_{1},a_{2},\ldots,a_{n} \right)
\left(b_{1},b_{2},\ldots,b_{n}  \right) =
\left( a_{1}b_{1},a_{2}b_{2},\ldots,a_{n}b_{n} \right) 
\end{equation*}
\vspace*{\stretch{1}}
\end{flashcard}

\begin{flashcard}[Definition]{internal direct product}
\vspace*{\stretch{1}}
A group $G$ is said to be the \textit{internal direct product}
of its normal subgroups $N_{1},N_{2},\ldots,N_{n}$ if every element of
$G$ has a unique representation, that is, if $a \in G$ then:
\begin{equation*}
a = a_{1},a_{2},\ldots,a_{n}  \text{ where each } a_{i} \in N_{i}
\end{equation*}
\vspace*{\stretch{1}}
\end{flashcard}

\begin{flashcard}[Lemma]{intersection of normal subgroups when the group
is an internal direct product}
\vspace*{\stretch{1}}
If $G$ is the internal direct product of its normal subgroups
$N_{1},N_{2},\ldots,N_{n}$, then for $i \neq j, N_{i} \cap N_{j} = 
\left\lbrace e \right\rbrace $.
\vspace*{\stretch{1}}
\end{flashcard}

\begin{flashcard}[Theorem]{isomorphism between an external direct product
and an internal direct product}
\vspace*{\stretch{1}}
Let $G$ be a group with normal subgroups $N_{1},N_{2},\ldots,N_{n}$, then the
mapping:
\begin{equation*}
\psi : N_{1} \times N_{2} \times \cdots \times N_{n} \mapsto G
\end{equation*}
defined by
\begin{equation*}
\psi((a_{1},a_{2},\ldots,a_{n})) = a_{1}a_{2}\cdots a_{n}
\end{equation*}
is an isomorphism iff $G$ is the internal direct product of 
$N_{1},N_{2},\ldots,N_{n}$.
\vspace*{\stretch{1}}
\end{flashcard}

\begin{flashcard}[Theorem]{fundamental theorem on finite abelian groups}
\vspace*{\stretch{1}}
A finite abelian group is the direct product of cyclic groups. 
\vspace*{\stretch{1}}
\end{flashcard}

\begin{flashcard}[Definition]{centralizer of an element}
\vspace*{\stretch{1}}
If $G$ is a group and $a \in G$, then the \textit{centralizer} of $a$ in $G$ is the
set of all elements in $G$ that commute with $a$.
\begin{equation*}
C(a) = \left\lbrace g \in G \mid ga = ag \right\rbrace 
\end{equation*}
\vspace*{\stretch{1}}
\end{flashcard}

\begin{flashcard}[Lemma]{the centralizer forms a subgroup}
\vspace*{\stretch{1}}
If $a \in G$, then $C(a)$ is a subgroup of $G$.
\vspace*{\stretch{1}}
\end{flashcard}

\begin{flashcard}[Theorem]{number of distinct conjugates of an element}
\vspace*{\stretch{1}}
Let $G$ be a finite group and $a \in G$, then the number of distinct
conjugates of $a$ in $G$ is $[G : C(a)]$ (the index of $C(a)$ in $G$).
\vspace*{\stretch{1}}
\end{flashcard}

\begin{flashcard}[Theorem]{the class equation}
\vspace*{\stretch{1}}
\begin{equation*}
|G| = |Z(G)| + \sum_{a \not{\in} Z(G)}[G : C(a)]
\end{equation*}
\vspace*{\stretch{1}}
\end{flashcard}

\begin{flashcard}[Theorem]{groups of order $p^{n}$}
\vspace*{\stretch{1}}
If $G$ is a group of order $p^{n}$, ($p$ prime) then $Z(G)$ is
non--trivial, i.e. there exists at least one element other than
the identity that commutes with all other elements of $G$.
\vspace*{\stretch{1}}
\end{flashcard}

\begin{flashcard}[Theorem]{groups of order $p^{2}$}
\vspace*{\stretch{1}}
If $G$ is a group of order $p^{2}$ ($p$ prime), then $G$ is abelian.
\vspace*{\stretch{1}}
\end{flashcard}

\begin{flashcard}[Theorem]{groups of order $p^{n}$ contain a normal 
subgroup}
\vspace*{\stretch{1}}
If $G$ is a group of order $p^{n}$ ($p$ prime), then $G$ contains
a normal subgroup of order $p^{n-1}$.
\vspace*{\stretch{1}}
\end{flashcard}

\begin{flashcard}[Definition]{$p$--Sylow group}
\vspace*{\stretch{1}}
If $G$ is a group of order $p^{n}m$ where $p$ is prime and 
$p \not{\mid} \: m$, then $G$ is a $p$--Sylow group.
\vspace*{\stretch{1}}
\end{flashcard}

\begin{flashcard}[Theorem]{Sylow's theorem (part 1)}
\vspace*{\stretch{1}}
If $G$ is a $p$--Sylow group ($|G| = p^{n}m$),
then $G$ has a subgroup of order $p^{n}$.
\vspace*{\stretch{1}}
\end{flashcard}

\begin{flashcard}[Theorem]{Sylow's theorem (part 2)}
\vspace*{\stretch{1}}
If $G$ is a $p$--Sylow group ($|G| = p^{n}m$), then any 
two subgroups of the same order are conjugate.  For example,
if $P$ and $Q$ are subgroups of $G$ where $|P|=|Q|=p^{n}$ then
\begin{equation*}
P = x^{-1}Qx \quad \text{ for some } x \in G
\end{equation*}
\vspace*{\stretch{1}}
\end{flashcard}

\begin{flashcard}[Theorem]{Sylow's theorem (part 3)}
\vspace*{\stretch{1}}
If $G$ is a $p$--Sylow group ($|G| = p^{n}m$),
then the number of subgroups of order $p^{n}$ in $G$ is
of the form $1 + kp$ and divides $|G|$.
\vspace*{\stretch{1}}
\end{flashcard}

\Card{Poset}{
    Partially ordered set: a set $S$ and relation $\leq$ where
    \items{
        \item $a \leq a$ (reflexive)
        \item $a \leq b$ and $b \leq a \Rightarrow a=b$ (antisymmetry)
        \item $a \leq b$ and $b \leq c \Rightarrow a\leq c$ (transitivity)
    } for all $a, b, c \in S$.

    A preorder with the addition of the antisymmetry condition.
}

\end{document}
