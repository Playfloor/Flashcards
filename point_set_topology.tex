\documentclass[avery5371,grid]{flashcards}

\usepackage{amsmath}
\usepackage{amsfonts}
\usepackage{amssymb}
\usepackage{ccicons}
\usepackage{mathrsfs}
\usepackage{url}

\newcommand{\N}{\mathbb{N}}
\newcommand{\Z}{\mathbb{Z}}
\newcommand{\Q}{\mathbb{Q}}
\newcommand{\R}{\mathbb{R}}
\newcommand{\st}{\textrm{ such that }}
\renewcommand{\le}{\leqslant}
\newcommand{\set}[2]{\ensuremath{\left\{ #1 \, : \, #2 \right\}}}
\newcommand{\presentation}[2]{\ensuremath{\left< #1 \, : \, #2 \right>}}

\newcommand{\defn}[1]{\textbf{#1}}

\cardfrontstyle[\large\slshape]{headings}
\cardbackstyle{empty}
\cardfrontfoot{Representation Theory}

\begin{document}

%%%%%%%%%%%%%%%%%%%%%%%%%%%%%%%%%%%%%%%%%%%%%%%%%%%%%%%%%%%%%%%%%%%%%%%%%%%%%%%
\begin{flashcard}[Copying]
  { Flash Cards for the Book:

    \begin{center}
      ``Representations and Characters of Groups'' \\
      by Gordon James and Martin Liebeck
    \end{center}
  }
  \vspace*{\stretch{1}}
  \copyright\ 2017 Jason Underdown \\

  These flash cards and the accompanying \LaTeX \, source code are
  licensed under a
  \begin{center}
    Creative Commons \\
    Attribution--NonCommercial--ShareAlike 4.0 International License \\
    \ccbyncsa
  \end{center}
  For more information: \url{creativecommons.org}
\vspace*{\stretch{1}}
\end{flashcard}

\begin{flashcard}[Definition]{group}
  \vspace*{\stretch{1}}

  A \defn{group} consists of a set $G$, together with a rule for
  combining any two elements $g, h \in G$ to form another element of
  $G$ satisfying:
  \begin{enumerate}
  \item $\forall g,h,k \in G, (gh)k = g(hk)$
  \item $\exists e \in G \st \forall g \in G, eg=ge=g$
  \item $\forall g \in G, \exists g^{-1} \in G \st gg^{-1} = g^{-1}g = e$
  \end{enumerate}

  \vspace*{\stretch{1}}
\end{flashcard}

\begin{flashcard}[Definition]{subgroup}
\vspace*{\stretch{1}}

Let $G$ be a group. A subset $H$ of $G$ is a \defn{subgroup} if $H$ is
itself a group under the operation inherited from $G$.
\[
  H \le G
\]

\vspace*{\stretch{1}}
\end{flashcard}

\begin{flashcard}[Definition]{dihedral group $D_{2n}$}
  \vspace*{\stretch{1}}
  \[
    D_{2n} = \presentation{a,b}{a^n=1, b^2=1, b^{-1}ab=a^{-1}}
  \]

  \vspace*{\stretch{1}}
\end{flashcard}

\begin{flashcard}[Definition]{cyclic group $C_n$}
  \vspace*{\stretch{1}}

  \[
    C_n = \left\{ 1, a, a^2, \ldots, a^{n-1} \right\}
  \]
  \[
    C_n = \presentation{a}{a^n=1}
  \]

  \vspace*{\stretch{1}}
\end{flashcard}

\begin{flashcard}[Definition]{quaternion group $Q_8$}
  \vspace*{\stretch{1}}

  \[
    Q_8 = \presentation{a,b}{a^4 = 1, a^2 = b^2, b^{-1}ab = a^{-1}}
  \]

  \vspace*{\stretch{1}}
\end{flashcard}

\begin{flashcard}[Definition]{alternating group $A_n$}
  \vspace*{\stretch{1}}
  \[
    A_n = \left\{ g \in S_n : g \text{ is an even permutation} \right\}
  \]
  Recall that every permutation $g \in S_n$ can be expressed as a
  product of transpositions. An \defn{even} permutation has an even
  number of transpositions, and an \defn{odd} permutation has an odd
  number of transpositions.

  \vspace*{\stretch{1}}
\end{flashcard}

\begin{flashcard}[Definition]{direct product}
  \vspace*{\stretch{1}}

  Let $G$ and $H$ be groups, consider
  \[
    G \times H = \set{(g,h)}{g \in G \text{ and } h \in H}.
  \]
  Define a product operation on $G \times H$ by
  \[
    (g,h)(g',h') = (gg', hh').
  \]
  The group $G \times H$ is called the \defn{direct product} of $G$ and
  $H$.

  \vspace*{\stretch{1}}
\end{flashcard}

\begin{flashcard}[]{}
  \vspace*{\stretch{1}}




  \vspace*{\stretch{1}}
\end{flashcard}

\begin{flashcard}[]{}
  \vspace*{\stretch{1}}




  \vspace*{\stretch{1}}
\end{flashcard}

%%%%%%%%%%%%%%%%%%%%%%%%%%%%%%%%%%%%%%%%%%%%%%%%%%%%%%%%%%%%%%%%%%%%%%%%%%%%%%%

\begin{flashcard}[]{}
  \vspace*{\stretch{1}}



  \vspace*{\stretch{1}}
\end{flashcard}

\begin{flashcard}[]{}
\vspace*{\stretch{1}}
\vspace*{\stretch{1}}
\end{flashcard}

\begin{flashcard}[]{}
\vspace*{\stretch{1}}
\vspace*{\stretch{1}}
\end{flashcard}

\begin{flashcard}[]{}
\vspace*{\stretch{1}}
\vspace*{\stretch{1}}
\end{flashcard}

\begin{flashcard}[]{}
\vspace*{\stretch{1}}
\vspace*{\stretch{1}}
\end{flashcard}

\begin{flashcard}[]{}
\vspace*{\stretch{1}}
\vspace*{\stretch{1}}
\end{flashcard}

\begin{flashcard}[]{}
\vspace*{\stretch{1}}
\vspace*{\stretch{1}}
\end{flashcard}

\begin{flashcard}[]{}
\vspace*{\stretch{1}}
\vspace*{\stretch{1}}
\end{flashcard}

\begin{flashcard}[]{}
\vspace*{\stretch{1}}
\vspace*{\stretch{1}}
\end{flashcard}

\begin{flashcard}[]{}
\vspace*{\stretch{1}}
\vspace*{\stretch{1}}
\end{flashcard}

%%%%%%%%%%%%%%%%%%%%%%%%%%%%%%%%%%%%%%%%%%%%%%%%%%%%%%%%%%%%%%%%%%%%%%%%%%%%%%%

\begin{flashcard}[]{}
\vspace*{\stretch{1}}
\vspace*{\stretch{1}}
\end{flashcard}

\begin{flashcard}[]{}
\vspace*{\stretch{1}}
\vspace*{\stretch{1}}
\end{flashcard}

\begin{flashcard}[]{}
\vspace*{\stretch{1}}
\vspace*{\stretch{1}}
\end{flashcard}

\begin{flashcard}[]{}
\vspace*{\stretch{1}}
\vspace*{\stretch{1}}
\end{flashcard}

\begin{flashcard}[]{}
\vspace*{\stretch{1}}
\vspace*{\stretch{1}}
\end{flashcard}

\begin{flashcard}[]{}
\vspace*{\stretch{1}}
\vspace*{\stretch{1}}
\end{flashcard}

\begin{flashcard}[]{}
\vspace*{\stretch{1}}
\vspace*{\stretch{1}}
\end{flashcard}

\begin{flashcard}[]{}
\vspace*{\stretch{1}}
\vspace*{\stretch{1}}
\end{flashcard}

\begin{flashcard}[]{}
\vspace*{\stretch{1}}
\vspace*{\stretch{1}}
\end{flashcard}

\begin{flashcard}[]{}
\vspace*{\stretch{1}}
\vspace*{\stretch{1}}
\end{flashcard}

%%%%%%%%%%%%%%%%%%%%%%%%%%%%%%%%%%%%%%%%%%%%%%%%%%%%%%%%%%%%%%%%%%%%%%%%%%%%%%%



\end{document}
