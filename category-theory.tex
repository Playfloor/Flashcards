\documentclass[onecard,frame]{flashcards}

\cardfrontstyle[\large\slshape]{headings}
\cardbackstyle{empty}
\cardfrontfoot{Category Theory}

%\usepackage{amsfonts}
\usepackage{amssymb,amsmath,tikz-cd}
\usepackage[T1]{fontenc}

\begin{document}

\input{tex.macros}

\long\def\Comdiag#1{
\begin{center}
\begin{tikzcd}[ampersand replacement=\&]
#1
\end{tikzcd}
\end{center}
}

\Card{Intro}{Author, copyleft}{Ben Klemens, CC BY/SA/NC}

\Dcard{Domain of $f$}{ For $f:X\to Y$, $X$.}
\Dcard{Codomain of $f$}{ For $f:X\to Y$, $Y$.}

\Dcard{Composition of $f$ and $g$}{
As a diagram,
\Comdiag{
X \arrow[r,"f"] \&Y \arrow[r, "g"] \& Z
}

Written $g \circ f: X\to Z$.
}

\Dcard{$Hom_{\bf \rm set}(X,Y)$}{The set of functions $X\to Y$}

\Dcard{Equality of $f$ and $g$}{
$f:X\to Y$ and $g:X\to Y$
are equal iff $f(x)=g(x), \forall x \in X$
}

\Dcard{Identity function on $X$}{
${\rm id}_X:X\to X$ such that ${\rm id}_X(x) = x, \forall x\in X$.
}

\Dcard{Isomorpism}{
A one-to-one correspondence.

\Comdiag{
    X \arrow[r,"\cong"] \&Y
}
There exists a function $g:Y\to X$ such that
$g \circ f = {\rm id}_X$ and
$f \circ g = {\rm id}_Y$.
}

\Card{Characteristics}{Properties of an isomorphism}{
\begin{itemize}
\item Reflexive: Any set is isomorphic to itself.
\item Commutative: if $f:A\to B$ has inverse $g:B\to A$, then $g$ has inverse $f$.
\item Transitive: $A$ isomorphic to $B$ and $B$ isomorphic to $C$ $\Rightarrow$ $A$ is
isomorphics to $C$.
\end{itemize}
}

\Card{Notation}{$\underline n$}{The set $\{1, 2,\dots n\}$}

\Dcard{Cardinality of finite set $X$}{$|X| = n$ iff

\Comdiag{
    X \arrow[r,"\cong"] \&\underline n
}
}

\Card{Lemma}{Cardinality of $A$ and $B$  given \begin{tikzcd}[ampersand replacement=\&]
    A \arrow[r,"\cong"] \&B
\end{tikzcd}
}{|A| = |B|}

\Dcard{Injective}{
    For $F:A\to B$, $f(x) = f(y)$ only if $x=y$: points map to different values.}


\Dcard{A diagram commutes}{
All paths from top left to bottom right are equal. E.g.,

\Comdiag{
A \ar[r, "f"] \ar[dr,"h"'] \& B\ar[d, "g"]\\
        \& C
}

commutes iff $g\circ f = h$.
}


\Dcard{Epimorphism}{
    $F$ is {\em epic} or an {\em epi} iff $g_1 f = g_2 f \Rightarrow g_1 = g_2$.

\Comdiag{
X \arrow[r,"f"] \&Y \arrow[r, shift left,"g_1"] \arrow[r,"g_2"'] \& Z
}


Analagous to an injective function.
}

\Card{Definition}{Monorphism}{
    $F$ is {\em monic} or {\em mono} iff $f g_1 = f g_2 \Rightarrow g_1 = g_2$.

\Comdiag{
Z \arrow[r, shift left,"g_1"] \arrow[r,"g_2"'] \&Y \arrow[r,"f"] \& X
}

}

\Card{Definition}{Isomorphism}{
    A mapping that is epic and monic.
}

\Card{Definition}{Group}{
A set with a binary operation that is

\begin{itemize}
\item    closed
\item    associative,
\item    has an identity, and
\item    has an inverse
\end{itemize}
}

\Dcard{Closed mapping}{For any element $x\in S$, $f(x)\in S$.}

\Dcard{Associative mapping}{A mapping $f:S\times S$ is
associative iff $f(s_1, f(s_2, s_3)) = f(f(s_1, s_2), s_3)$.
}

\Dcard{Magma}{A set with a closed mapping.}

\Dcard{Semigroup}{A set with a closed, associative mapping.

A magma whose operation is also associative.}

\Dcard{Monoid}{A set with an operation that is closed and associative,
and an identity element where $f(x)=x$.

A semigroup with an identity.}

\Dcard{Hom set of $A$, $B$}{The set of mappings between $A$ and $B$}

\end{document}
