\documentclass[avery5371]{flashcards}

\cardfrontstyle[\large\slshape]{headings}
\cardbackstyle{empty}

\begin{document}

\cardfrontfoot{Mechanics}

%% Flashcards based on the book "Physics for Scientists and Engineers with Modern Physics"
%% 6th edition by Serway and Jewett

%% Chapter 1

\begin{flashcard}[Chapter 1]{Density}
\bigskip
\bigskip
\begin{displaymath}
\rho \equiv \frac{m}{V}
\end{displaymath}
\end{flashcard}

\begin{flashcard}[Chapter 1]{Significant Figures:  Multiplication}
\bigskip
\bigskip
\begin{quote}
When multiplying several quantities, the number of significant figures in the final answer is the same as the number of significant figures in the quantity having the lowest number of significant figures.  The same rule applies to division.
\end{quote}
\hfill
\end{flashcard}

\begin{flashcard}[Chapter 1]{Significant Figures:  Addition}
\bigskip
\bigskip
\begin{quote}
When numbers are added or subtracted, the number of decimal places in the result should equal the smallest number of decimal places of any term in the sum.
\end{quote}
\hfill
\end{flashcard}

%% Chapter 2

\begin{flashcard}[Chapter 2]{Displacement}
\bigskip
\begin{displaymath}
\Delta x \equiv x_{f} - x_{i}
\end{displaymath}
\begin{center}
\textbf{or} \\
\medskip
Displacement = area under the $v_{x}$--$t$ graph
\end{center}
\end{flashcard}

\begin{flashcard}[Chapter 2]{Average velocity}
\bigskip
\bigskip
\begin{displaymath}
\bar{v}_{x} \equiv \frac{\Delta x}{\Delta t} 
\end{displaymath}
\end{flashcard}

\begin{flashcard}[Chapter 2]{Average speed}
\bigskip
\bigskip
\begin{displaymath}
\textrm{Average speed} = \frac{\textrm{total distance}}{\textrm{total time}}
\end{displaymath}
\end{flashcard}

\begin{flashcard}[Chapter 2]{Instantaneous velocity}
\bigskip
\bigskip
\begin{displaymath}
v_{x} \equiv \lim_{\Delta t \rightarrow 0} \frac{\Delta x}{\Delta t} = \frac{dx}{dt}
\end{displaymath}
\end{flashcard}

\begin{flashcard}[Chapter 2]{Average acceleration}
\bigskip
\bigskip
\begin{displaymath}
\bar{a}_{x} \equiv \frac{\Delta v_{x}}{\Delta t} = \frac{v_{xf} - v_{xi}}{t_{f} - t_{i}}
\end{displaymath}
\end{flashcard}

\begin{flashcard}[Chapter 2]{Instantaneous acceleration}
\bigskip
\bigskip
\begin{displaymath}
a_{x} \equiv \lim_{\Delta t \rightarrow 0} \frac{\Delta v_{x}}{\Delta t} = \frac{dv_{x}}{dt}
\end{displaymath}
\end{flashcard}

\begin{flashcard}[Chapter 2]{Velocity as a function of time}
\bigskip
\begin{displaymath}
v_{xf} = v_{xi} + a_{x}t
\end{displaymath}
\begin{center}
(constant acceleration)
\end{center}
\end{flashcard}

\begin{flashcard}[Chapter 2]{Position as a function of velocity and time}
\bigskip
\begin{displaymath}
x_{f} = x_{i} + \frac{1}{2}\left( v_{xi} + v_{xf}\right)t
\end{displaymath}
\begin{center}
(constant acceleration)
\end{center}
\end{flashcard}

\begin{flashcard}[Chapter 2]{Position as a function of time}
\bigskip
\begin{displaymath}
x_{f} = x_{i} + v_{xi}t + \frac{1}{2}a_{x}t
\end{displaymath}
\begin{center}
(constant acceleration)
\end{center}
\end{flashcard}

\begin{flashcard}[Chapter 2]{Velocity as a function of position}
\bigskip
\begin{displaymath}
v_{xf}^{2} = v_{xi}^{2} + 2a_{x}\left( x_{f} - x_{i}\right) 
\end{displaymath}
\begin{center}
(constant acceleration)
\end{center}
\end{flashcard}

%% Chapter 3

\begin{flashcard}[Chapter 3]{Polar $\Longrightarrow$ Cartesian}
\bigskip
\bigskip
\begin{eqnarray}
x &=& r\cos \theta \nonumber\\
y &=& r\sin \theta \nonumber
\end{eqnarray}
\end{flashcard}

\begin{flashcard}[Chapter 3]{Cartesian $\Longrightarrow$ Polar}
\bigskip
\bigskip
\begin{eqnarray}
r &=& \sqrt{x^{2} + y^{2}} \nonumber\\
\tan \theta &=& \frac{y}{x} \nonumber
\end{eqnarray}
\end{flashcard}

\begin{flashcard}[Chapter 3]{Scalar quantity}
\bigskip
\bigskip
\begin{quote}
A value with magnitude only and \textbf{no} associated direction
\end{quote}
\hfill
\end{flashcard}

\begin{flashcard}[Chapter 3]{Vector quantity}
\bigskip
\bigskip
\begin{quote}
A value that has both magnitude and direction
\end{quote}
\hfill
\end{flashcard}

%% Chapter 4

\begin{flashcard}[Chapter 4]{Velocity vector as a function of time}
\bigskip
\bigskip
\begin{displaymath}
\mathbf{v}_{f} = \mathbf{v}_{i} + \mathbf{a} t
\end{displaymath}
\end{flashcard}

\begin{flashcard}[Chapter 4]{Position vector as a function of time}
\bigskip
\bigskip
\begin{displaymath}
\mathbf{r}_{f} = \mathbf{r}_{i} + \mathbf{v}_{i}t + \frac{1}{2} \mathbf{a}t^{2}
\end{displaymath}
\end{flashcard}

\begin{flashcard}[Chapter 4]{Centripetal acceleration}
\bigskip
\bigskip
\begin{displaymath}
a_{c} = \frac{v^{2}}{r}
\end{displaymath}
\end{flashcard}

\begin{flashcard}[Chapter 4]{Period of circular motion}
\bigskip
\bigskip
\begin{displaymath}
T \equiv \frac{2 \pi r}{v}
\end{displaymath}
\end{flashcard}

\begin{flashcard}[Chapter 4]{Total acceleration}
\bigskip
\bigskip
\begin{displaymath}
\mathbf{a} = \mathbf{a}_{t} + \mathbf{a}_{r} = \frac{d \mid \mathbf{v} \mid}{dt} \hat{\mathbf{\theta}} - \frac{v^{2}}{r} \hat{\mathbf{r}}
\end{displaymath}
\end{flashcard}

\begin{flashcard}[Chapter 4]{Galilean Transformation}
\bigskip
\bigskip
\begin{eqnarray}
\mathbf{r}^{\prime} &=& \mathbf{r} - \mathbf{v}_{0}t \nonumber\\
\mathbf{v}^{\prime} &=& \mathbf{v} - \mathbf{v}_{0} \nonumber
\end{eqnarray}
\end{flashcard}

%% Chapter 5

\begin{flashcard}[Chapter 5]{Newton's First Law}
\smallskip
\begin{quote}
In the absence of external forces, when viewed from an inertial reference frame, an object at rest remains at rest and an object in motion continues in motion with a constant velocity (that is, with a constant speed in a a straight line).
\\
\smallskip
\textbf{When no force acts on an object, the acceleration of the object is zero}
\end{quote}
\hfill
\end{flashcard}


\begin{flashcard}[Chapter 5]{Newton's Second Law}
\bigskip
\begin{quote}
When viewed from an inertial reference frame, the acceleration of an object is directly proportional to the net force acting on it and inversely proportional to its mass.
\end{quote}
\begin{displaymath}
\sum \mathbf{F} = m\mathbf{a}
\end{displaymath}
\hfill
\end{flashcard}


\begin{flashcard}[Chapter 5]{Newton's Third Law}
\bigskip
\begin{quote}
If two objects interact, the force $\mathbf{F}_{12}$ exerted by object 1 on object 2 is equal in magnitude and opposite in direction to the force $\mathbf{F}_{21}$ exerted by object 2 on object 1:
\end{quote}
\begin{displaymath}
\mathbf{F}_{12} = -\mathbf{F}_{21}
\end{displaymath}
\hfill
\end{flashcard}

%% Chapter 6

\begin{flashcard}[Chapter 6]{Force causing centripetal acceleration}
\bigskip
\bigskip
\begin{displaymath}
\sum \mathbf{F} = ma_{c} = m\frac{v^{2}}{r}
\end{displaymath}
\end{flashcard}

\begin{flashcard}[Chapter 6]{Nonuniform circular motion}
\bigskip
\bigskip
\begin{displaymath}
\sum \mathbf{F} = \sum \mathbf{F}_{r} + \sum \mathbf{F}_{t}
\end{displaymath}
\end{flashcard}

%% Chapter 7

\begin{flashcard}[Chapter 7]{Scalar, dot or inner product}
\bigskip
\bigskip
\begin{eqnarray}
\mathbf{A}\cdot \mathbf{B} &=& AB \cos \theta \nonumber \\
\mathbf{A}\cdot \mathbf{B} &=& A_{x}B_{x} + A_{y}B_{y} + A_{z}B_{z} \nonumber
\end{eqnarray}
\end{flashcard}


\begin{flashcard}[Chapter 7]{Work done by a constant force}
\bigskip
\bigskip
\begin{displaymath}
W \equiv F\Delta r \cos \theta
\end{displaymath}
\end{flashcard}

\begin{flashcard}[Chapter 7]{Work done by a varying force}
\bigskip
\bigskip
\begin{displaymath}
W = \int_{x_{i}}^{x_{f}} F_{x}dx
\end{displaymath}
\end{flashcard}

\begin{flashcard}[Chapter 7]{Spring force}
\bigskip
\bigskip
\begin{displaymath}
F_{s} = -kx
\end{displaymath}
\end{flashcard}


\end{document}
