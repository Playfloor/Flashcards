\documentclass[avery5371,grid]{flashcards}

\newcommand{\deriv}[2]{\frac{\mathrm{d}#1}{\mathrm{d}#2}}
\newcommand{\pderiv}[2]{\frac{\partial#1}{\partial#2}}
\newcommand{\dQ}[0]{d^{\prime}\!Q}
\newcommand{\dW}[0]{d^{\prime}\!W}

\cardfrontstyle[\large\slshape]{headings}
\cardbackstyle{empty}

\begin{document}
\cardfrontfoot{Thermodynamics}

\begin{flashcard}[Equation]{Ideal Gas Law}
  \vspace*{\stretch{1}}
  \begin{center}
    \begin{displaymath}
      Pv = nRT
    \end{displaymath}
  \end{center}
  \vspace*{\stretch{1}}
\end{flashcard}

\begin{flashcard}[Equation]{Van der Waals Equation}
  \vspace*{\stretch{1}}
  \begin{center}
    \begin{displaymath}
      \left(P+\frac{a}{v^2}\right)\left(v-b\right) = RT
    \end{displaymath}
  \end{center}
  \vspace*{\stretch{1}}
\end{flashcard}

\begin{flashcard}[Definition]{Coefficient of Volume Expansion\\$\beta$}
  \vspace*{\stretch{1}}
  \begin{center}
    \begin{displaymath}
      \beta = \frac{1}{V}{\left(\pderiv{V}{T}\right)}_P
    \end{displaymath}
  \end{center}
  \vspace*{\stretch{1}}
\end{flashcard}

\begin{flashcard}[Definition]{Isothermal Compressibility\\$\kappa$}
  \vspace*{\stretch{1}}
  \begin{center}
    \begin{displaymath}
      \kappa= -\frac{1}{V}\left(\pderiv{V}{P}\right)_T
    \end{displaymath}
  \end{center}
  \vspace*{\stretch{1}}
\end{flashcard}

\begin{flashcard}[Equation]{Volume Differential\\$dV$}
  \vspace*{\stretch{1}}
  \begin{center}
    \begin{displaymath}
      dV = {\left(\pderiv{V}{T}\right)}_P\!\!dT  + {\left(\pderiv{V}{P}\right)}_T\!\!dP
    \end{displaymath}
  \end{center}
  \vspace*{\stretch{1}}
\end{flashcard}

\begin{flashcard}[Definition]{Exact Differential}
  \vspace*{\stretch{1}}
  \begin{tiny}
  The following two properties are equivalent ways of determining exactness:\\
  1. Mixed second order partial derivatives are equal e.g.:
  \begin{displaymath}
    \frac{\partial^2 V}{\partial P \partial T} = 
    \frac{\partial^2 V}{\partial T \partial P}
  \end{displaymath}
  2. Integral is independent of path
  \begin{displaymath}
    \int_{V_1}^{V_2} dV = V_1 - V_2 \qquad \oint dV = 0
  \end{displaymath}
  A quantity whose differential is \emph{not} exact is not a thermodynamic property.
  \end{tiny}
  \vspace*{\stretch{1}}
\end{flashcard}

\begin{flashcard}[Law]{First Law of Thermodynamics}
  \vspace*{\stretch{1}}
  \begin{center}
    \begin{displaymath}
      \begin{array}{ll}
	\Delta U = & Q - W\\
	 & \\
	dU = &\dQ - \dW
      \end{array}
    \end{displaymath}
    \medskip
    (Where the primes denote inexact differentials)
  \end{center}
  \vspace*{\stretch{1}}
\end{flashcard}

\begin{flashcard}[Definition]{Enthalpy}
  \vspace*{\stretch{1}}
  \begin{center}
    \begin{displaymath}
      H = U + PV
    \end{displaymath}
  \end{center}
  \vspace*{\stretch{1}}
\end{flashcard}

\begin{flashcard}[Definition]{Heat Capacity}
  \vspace*{\stretch{1}}
  \begin{center}
    \begin{displaymath}
      C = \lim_{\Delta T\to0} \frac{Q}{\Delta T} = \frac{\dQ}{dT}
    \end{displaymath}
    \begin{displaymath}
      Q = C(T_2 - T_1) = nc(T_2 - T_1)
    \end{displaymath}
  \end{center}
  \vspace*{\stretch{1}}
\end{flashcard}

\begin{flashcard}[Equation]{Thermodynamic Potentials}
  \vspace*{\stretch{1}}
  \begin{center}
    \begin{tabular}{rc}
       & \begin{math}-TS\end{math} \\
       & \begin{math}\longrightarrow\end{math} \\
      \begin{math}+PV \downarrow\end{math} &
      {
	\begin{tabular}{|c|c|}
	  \hline
	  U & F \\
	  \hline
	  H & G \\
	  \hline
	\end{tabular}
      } \\
    \end{tabular}
  \end{center}
  \vspace*{\stretch{1}}
\end{flashcard}

\end{document}
