\documentclass[avery5371,grid]{flashcards}

\cardfrontstyle[\large\slshape]{headings}
\cardbackstyle{empty}
\cardfrontfoot{Topology}

\usepackage{amsmath}
\usepackage{amsfonts}
\usepackage{amssymb}
\usepackage{mathrsfs}

\newcommand{\N}{\mathbb{N}}
\newcommand{\Z}{\mathbb{Z}}
\newcommand{\Q}{\mathbb{Q}}
\newcommand{\R}{\mathbb{R}}
\newcommand{\st}{\textrm{ such that }}
\newcommand{\setst}{\; | \;}

\begin{document}

\begin{flashcard}[Copyright \& License]{Copyright \copyright \,
2007 Jason Underdown \\ Some rights reserved.}
\vspace*{\stretch{1}}
These flashcards and the accompanying \LaTeX \, source code are licensed
under a Creative Commons Attribution--NonCommercial--ShareAlike 2.5 License.  
For more information, see creativecommons.org.  You can contact the author at:
\begin{center}
\begin{small}\tt jasonu at physics utah edu\end{small}
\end{center}
\vspace*{\stretch{1}}
\end{flashcard}

\begin{flashcard}[Definition]{metric space}
\vspace*{\stretch{1}}
A \textbf{metric space} $(X,d)$ is a set $X$ and a function
\mbox{$d: X \times X \rightarrow \R$} satisfying $\forall \; x,y,z \in X$
\begin{enumerate}
 \item $d(x,y) \geq 0$
 \item $d(x,y) = 0 \Leftrightarrow x=y$
 \item $d(x,y) = d(y,x)$
 \item $d(x,z) \leq d(x,y) + d(y,z)$
\end{enumerate}
\vspace*{\stretch{1}}
\end{flashcard}

\begin{flashcard}[Definition]{subspace}
\vspace*{\stretch{1}}
If $(X,d)$ is a metric space, and $A \subset X$ then
$(A,d|_{A \times A})$ is a metric space and is called
a \mbox{\textbf{subspace}} of $(X,d)$.
\vspace*{\stretch{1}}
\end{flashcard}

\begin{flashcard}[Definition]{isometry}
\vspace*{\stretch{1}}
Suppose $(X_1,d_1)$ and $(X_2,d_2)$ are metric spaces.  A function
$f: X_1 \rightarrow X_2$ is called an \mbox{\textbf{isometry}} if 
$f$ is one--to--one, onto and
\begin{equation*}
d_2(f(x),f(y)) = d_1(x,y) \quad \forall \; x,y \in X_1
\end{equation*}
\vspace*{\stretch{1}}
\end{flashcard}

\begin{flashcard}[Definition]{open set}
\vspace*{\stretch{1}}
Supposing $(X,d)$ is a metric space, then a subset $U \subset X$
is \textbf{open} iff
\begin{equation*}
\forall \; x \in U, \exists \; r>0 \st B(x,r) \subset U
\end{equation*}
\vspace*{\stretch{1}}
\end{flashcard}

\begin{flashcard}[Proposition]{open balls are open}
\vspace*{\stretch{1}}
If $(X,d)$ is a metric space, then for each $x \in X$ and for each
$r>0$, $B(x,r)$ is open in X.
\vspace*{\stretch{1}}
\end{flashcard}

\begin{flashcard}[Theorem]{unions and intersections of open sets}
\vspace*{\stretch{1}}
Let $(X,d)$ be a metric space and let $\left\{U_{\alpha}\right\}_{\alpha \in A}$ be any
collection of open sets in $(X,d)$, then
\begin{enumerate}
 \item $X,\varnothing$ are open.
 \item $\bigcup_{\alpha \in A} U_{\alpha}$ is open.
 \item Let $\left\{U_1, \ldots , U_n\right\}$ be a finite
 collection of open sets, then $\bigcap^{n}_{i=1}U_i$ is open.
\end{enumerate}
\vspace*{\stretch{1}}
\end{flashcard}

\begin{flashcard}[Definition]{closed set}
\vspace*{\stretch{1}}
Let $(X,d)$ be a metric space, $F \subset X$ is \mbox{\textbf{closed}} iff
$X - F$ is open.
\vspace*{\stretch{1}}
\end{flashcard}

\begin{flashcard}[Definition]{closed ball}
\vspace*{\stretch{1}}
A \mbox{\textbf{closed ball}} centered at $x$ of radius $r$ is denoted
$\overline{B}(x,r)$, and defined to be:
\begin{equation*}
\overline{B}(x,r) = \left\{ y \in X \setst d(x,y) \leq r \right\}
\end{equation*}
\vspace*{\stretch{1}}
\end{flashcard}

\begin{flashcard}[Proposition]{closed balls are closed sets}
\vspace*{\stretch{1}}
A closed ball $\overline{B}(x,r)$, is a closed set.
\vspace*{\stretch{1}}
\end{flashcard}

\begin{flashcard}[Theorem]{unions and intersections of closed sets}
\vspace*{\stretch{1}}
Let $(X,d)$ be a metric space and let $\left\{F_{\alpha}\right\}_{\alpha \in A}$ be any
collection of closed sets in $(X,d)$, then
\begin{enumerate}
 \item $X,\varnothing$ are closed.
 \item $\bigcap_{\alpha \in A} F_{\alpha}$ is closed.
 \item Let $\left\{F_1, \ldots , F_n\right\}$ be a finite collection of
 closed sets, then $\bigcup^{n}_{i=1}F_i$ is closed.
\end{enumerate}
\vspace*{\stretch{1}}
\end{flashcard}

\begin{flashcard}[Definition]{interior}
\vspace*{\stretch{1}}
Let $(X,d)$ be a metric space with $A \subset X$.  The \mbox{\textbf{interior}}
of $A$ denoted $A^{\circ}$  is defined to be:
\begin{equation*}
A^{\circ} = \left\{ x \in A \setst \exists \; r>0 \st B(x,r) \subset A \right\}
\end{equation*}
\vspace*{\stretch{1}}
\end{flashcard}

\begin{flashcard}[Definition]{closure}
\vspace*{\stretch{1}}
Let $(X,d)$ be a metric space with $A \subset X$.  The \mbox{\textbf{closure}}
of $A$ denoted $\overline{A}$  is defined to be:
\begin{equation*}
\overline{A} = \left\{ x \in X \setst \forall \; r>0, B(x,r) \cap A
\neq \varnothing\right\}
\end{equation*}
\vspace*{\stretch{1}}
\end{flashcard}

\begin{flashcard}[Definition]{exterior \& frontier}
\vspace*{\stretch{1}}
Let $(X,d)$ be a metric space with $A \subset X$.

\bigskip
The \mbox{\textbf{exterior}} of a set $A$ is defined to be $(X-A)^{\circ}$.

\bigskip
The \mbox{\textbf{frontier}} of a set $A$ is defined to be $\overline{A}-A^{\circ}$.
\vspace*{\stretch{1}}
\end{flashcard}

\begin{flashcard}[Definition]{distance from a point to a set}
\vspace*{\stretch{1}}
Suppose $(X,d)$ is a metric space with $A \subset X$ and $x \in X$.  We
define \textbf{the distance from $x$ to $A$} by
\begin{equation*}
d(x,A) = \textrm{inf} \left\{ d(x,y) \setst y \in A \right\}
\end{equation*} 
\vspace*{\stretch{1}}
\end{flashcard}

\begin{flashcard}[Definition]{limit of a sequence}
\vspace*{\stretch{1}}
Suppose $(X,d)$ is a metric space.  A sequence $\left\{ x_n \right\} \subset X$
has \mbox{\textbf{limit}} $x$, denoted 
$\lim_{n \rightarrow \infty}   \left\{ x_n \right\} = x$ iff
\begin{equation*}
\forall \; \varepsilon >0, \; \exists \; N \in \N \st
\end{equation*}
\begin{equation*}
n \geq N \Rightarrow x_n \in B(x,\varepsilon)
\end{equation*}
\vspace*{\stretch{1}}
\end{flashcard}

\begin{flashcard}[Definition]{Cauchy Sequence}
\vspace*{\stretch{1}}
Suppose $(X,d)$ is a metric space.
A sequence $\left\{ x_n \right\} \subset X$ is called a
\mbox{\textbf{Cauchy sequence}} iff
\begin{equation*}
\forall \; \varepsilon >0, \; \exists \; N \in \N \st
\end{equation*}
\begin{equation*}
m,n \geq N \Rightarrow d(x_m,x_n) < \varepsilon
\end{equation*}
\vspace*{\stretch{1}}
\end{flashcard}

\begin{flashcard}[Definition]{convergent sequence}
\vspace*{\stretch{1}}
A sequence $\left\{ x_n \right\}$ \mbox{\textbf{converges}} iff
$\lim \left\{ x_n \right\}$ exits.
\vspace*{\stretch{1}}
\end{flashcard}

\begin{flashcard}[Theorem]{convergence implies Cauchy}
\vspace*{\stretch{1}}
If a sequence $\left\{ x_n \right\}$ is convergent then it is Cauchy.
\vspace*{\stretch{1}}
\end{flashcard}

\begin{flashcard}[Definition]{complete metric space}
\vspace*{\stretch{1}}
A metric space $(X,d)$ is \mbox{\textbf{complete}} iff every Cauchy sequence
in $X$ is convergent.
\vspace*{\stretch{1}}
\end{flashcard}

\begin{flashcard}[Theorem]{limits are unique}
\vspace*{\stretch{1}}
If the limit of $\left\{ x_n \right\}$ exists, then that limit is unique.
\vspace*{\stretch{1}}
\end{flashcard}

\begin{flashcard}[Theorem]{distinct points have a radius of separation}
\vspace*{\stretch{1}}
Suppose $(X,d)$ is a metric space, and $x,y \in X$ with $x \neq y$, then
$\exists \; r > 0 \st B(x,r) \cap B(y,r) = \varnothing$
\vspace*{\stretch{1}}
\end{flashcard}

\begin{flashcard}[Definition]{continuous function}
\vspace*{\stretch{1}}
Suppose $(X_1,d_1), (X_2, d_2)$ are metric spaces.  A function
$f:X_1 \rightarrow X_2$ is \mbox{\textbf{continuous}} at $x \in X_1$ iff
\begin{equation*}
\forall \; \varepsilon >0, \; \exists \; \delta (x,\varepsilon) >0 \st 
\end{equation*}
\begin{equation*}
d_1(x,y) < \delta \Rightarrow d_2(f(x), f(y)) < \varepsilon
\end{equation*}
\vspace*{\stretch{1}}
\end{flashcard}

\begin{flashcard}[Definition]{continuous function (alternate definition)}
\vspace*{\stretch{1}}
Suppose $(X_1,d_1), (X_2, d_2)$ are metric spaces.  A function
$f:X_1 \rightarrow X_2$ is \mbox{\textbf{continuous}} on $X_1$ iff
\begin{equation*}
\forall \; x \in X_1, \; \forall \; \varepsilon >0, \; \exists \; \delta >0 \st 
\end{equation*}
\begin{equation*}
f(B(x,\delta)) \subset B(f(x), \varepsilon)
\end{equation*}
\vspace*{\stretch{1}}
\end{flashcard}

\begin{flashcard}[Definition]{Lipschitz function}
\vspace*{\stretch{1}}
Suppose $(X_1,d_1), (X_2, d_2)$ are metric spaces.  A function
$f:X_1 \rightarrow X_2$ is called \mbox{\textbf{Lipschitz}} iff
\begin{equation*}
\forall \; x,y \in X_1 \; \exists \; c>0 \st
\end{equation*}
\begin{equation*}
d_2(f(x),f(y)) \leq c d_1(x,y)
\end{equation*}
A Lipschitz function can be thought of as a ``bounded distortion.''
\vspace*{\stretch{1}}
\end{flashcard}

\begin{flashcard}[Theorem]{Lipschitz functions are uniformly continuous}
\vspace*{\stretch{1}}
If $f:X_1 \rightarrow X_2$ is Lipschitz on $X_1$, then $f$ is uniformly
continuous on $X_1$.
\vspace*{\stretch{1}}
\end{flashcard}

\begin{flashcard}[Definition]{bi-Lipschitz}
\vspace*{\stretch{1}}
Suppose $(X_1,d_1), (X_2, d_2)$ are metric spaces.  A function
$f:X_1 \rightarrow X_2$ is called \mbox{\textbf{bi-Lipschitz}} iff
\begin{equation*}
\forall \; x,y \in X_1 \; \exists \; c_1,c_2>0 \st
\end{equation*}
\begin{equation*}
c_1 d_1(x,y) \leq d_2(f(x),f(y)) \leq c_2 d_1(x,y)
\end{equation*}
\vspace*{\stretch{1}}
\end{flashcard}

\begin{flashcard}[Theorem]{$f$ continuous iff \\
the preimage of every open set is open}
\vspace*{\stretch{1}}
A function $f:X_1 \rightarrow X_2$ is continuous iff 
\begin{equation*}
\forall \; U \mbox{ open }\subset X_2 \Rightarrow
f^{-1}(U) \mbox{ open }\subset X_1
\end{equation*}
Or equivalently:
\begin{equation*}
\forall \; U \mbox{ closed }\subset X_2 \Rightarrow
f^{-1}(U) \mbox{ closed }\subset X_1
\end{equation*}
\vspace*{\stretch{1}}
\end{flashcard}

\begin{flashcard}[Theorem]{continuous functions and sequences}
\vspace*{\stretch{1}}
A function $f:(X_1,d_1) \rightarrow (X_2,d_2)$ is continuous iff
\begin{equation*}
\forall \; \mbox{ convergent sequences } \left\{ x_n \right\} \subset X_1,
\end{equation*}
\begin{equation*}
\lim_{n\rightarrow \infty} f(x_n) = f(\lim_{n\rightarrow \infty}
\left\{ x_n \right\})
\end{equation*}
\vspace*{\stretch{1}}
\end{flashcard}

\begin{flashcard}[Definition]{homeomorphism}
\vspace*{\stretch{1}}
A function $f:(X_1,d_1) \rightarrow (X_2,d_2)$ is called a
\mbox{\textbf{homeomorphism}} iff
\begin{enumerate}
 \item $f$ is continuous
 \item $f$ is 1-1 and onto
 \item $f^{-1}$ is continous
\end{enumerate}
\vspace*{\stretch{1}}
\end{flashcard}

\begin{flashcard}[Definition]{equivalent metrics}
\vspace*{\stretch{1}}
Two metrics $d_1$, $d_2$ are called \mbox{\textbf{equivalent}} iff
they have the same open sets.
\vspace*{\stretch{1}}
\end{flashcard}

\begin{flashcard}[Remark]{two metrics are equivalent iff \\
the identity map is a homeomorphism}
\vspace*{\stretch{1}}
Two metrics, $d_1$, $d_2$ are equivalent iff $id: (X,d_1) \rightarrow (X,d_2)$
is a homeomorphism.
\vspace*{\stretch{1}}
\end{flashcard}

\begin{flashcard}[Theorem]{composition of continuous functions \\
preserves continuity}
\vspace*{\stretch{1}}
Suppose $f:X_1 \rightarrow X_2$ and $g:X_2 \rightarrow X_3$.  If $f$
and $g$ are continuous then $g\circ f$ is continuous.
\vspace*{\stretch{1}}
\end{flashcard}

\begin{flashcard}[Definition]{homeomorphic spaces}
\vspace*{\stretch{1}}
Two metric spaces are \mbox{\textbf{homeomorphic}} iff there exists a
homeomorphism between them.
\vspace*{\stretch{1}}
\end{flashcard}

\begin{flashcard}[Definition]{topology}
\vspace*{\stretch{1}}
Suppose $X$ is a set.  A collection $\tau$ of subsets of $X$ is called
a \mbox{\textbf{topology}} on $X$ iff
\begin{enumerate}
 \item $X \in \tau$ and $\varnothing \in \tau$
 \item $U_{\alpha} \in \tau \;
 \mbox{ for } \alpha \in A \Rightarrow
 \displaystyle \bigcup_{\alpha \in A} U_{\alpha} \in \tau$
 \item $U_1, U_2, \ldots , U_n \in \tau \Rightarrow
 \displaystyle \bigcap_{i=1}^{\infty} U_i \in \tau$
\end{enumerate}
\vspace*{\stretch{1}}
\end{flashcard}

\begin{flashcard}[Definition]{topological space}
\vspace*{\stretch{1}}
A \mbox{\textbf{topological space}} $(X,\tau)$ is a set $X$ and a 
topology $\tau$ on $X$.
\vspace*{\stretch{1}}
\end{flashcard}

\end{document}
